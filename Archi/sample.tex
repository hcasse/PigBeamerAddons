\documentclass[handout, 10pt]{beamer}
\usetheme{Archi}
\usepackage[utf8]{inputenc}
\usepackage[english]{babel}

\usepackage{lipsum}
\usepackage{layout}

\title{Archi Theme Multiline}
\subtitle{PowerPoint Inspired Theme}
\date[2021]{2021}
\author{M. Self}
\institute{There Institute}
\titlegraphic{\includegraphics[width=20mm]{title.png}}

\newcommand{\toc}{
	\begin{frame}{Outline}
		\tableofcontents[currentsection]
	\end{frame}
}

\begin{document}

%\begin{frame}
% \layout
%\end{frame}

\frame{
	\maketitle
}


\section{Introduction}

\begin{frame}{A few words}
	\lipsum[1-1]
\end{frame}

\begin{frame}{Theme colors}
	\begin{center}
		\begin{tikzpicture}
			\node[myshadow, circle, fill=myorange, minimum width=1.2cm] (myorange) {};
			\node[below=0mm of myorange] {\footnotesize myorange};
			\node[myshadow, circle, fill=mydarkorange, minimum width=1.2cm, right=of myorange] (mydarkorange) {};
			\node[below=0mm of mydarkorange] {\footnotesize mydarkorange};
			\node[myshadow, circle, fill=myred, minimum width=1.2cm, right=of mydarkorange] (myred) {};
			\node[below=0mm of myred] {\footnotesize myred};
			\node[myshadow, circle, fill=mydarkred, minimum width=1.2cm, right=of myred] (mydarkred) {};
			\node[below=0mm of mydarkred] {\footnotesize mydarkred};
			\node[myshadow, circle, fill=mypink, minimum width=1.2cm, right=of mydarkred] (mypink) {};
			\node[below=0mm of mypink] {\footnotesize mypink};
		\end{tikzpicture}		

		\bigskip
		\begin{tikzpicture}
			\node[rectangle, draw, black, myshadow, fill=white, rounded corners] {myshadow tikz style};
		\end{tikzpicture}

		\bigskip
		\begin{tikzpicture}
			\node[myshadow, circle, fill=mybrown, minimum width=1.2cm] (mybrown) {};
			\node[myshadow, circle, fill=mymate, minimum width=1.2cm, right=of mybrown] (mymate) {};
			\node[myshadow, circle, fill=myblue, minimum width=1.2cm, right=of mymate] (myblue) {};
			\node[myshadow, circle, fill=mydarkblue, minimum width=1.2cm, right=of myblue] (mydarkblue) {};			
			\node[below=0mm of mybrown] {\footnotesize mybrown};
			\node[below=0mm of mymate] {\footnotesize mymate};
			\node[below=0mm of myblue] {\footnotesize myblue};
			\node[below=0mm of mydarkblue] {\footnotesize mydarkblue};
		\end{tikzpicture}
	\end{center}
\end{frame}


\section{First Section}


\begin{frame}{Itemize and Enumerated}
	This an itemize list:
	\begin{itemize}
		\item item 1
		\item item 2
		\item item 3
	\end{itemize}

	And an enumerate list:
	\begin{enumerate}
		\item item a
		\item item b
		\item item c
	\end{enumerate}
\end{frame}


\begin{frame}{Several paragraphs}
	This the first paragraph.

	Followed by another paragraph.

	And then a third one.
\end{frame}


\begin{frame}{Highlighting}

	This text contains an \alert{alert}!

	\begin{alertenv}
		This is an alert environment!
	\end{alertenv}

	\begin{structureenv}
		A structure environment.
	\end{structureenv}

	And finally a \structure{structure}.

\end{frame}

\begin{frame}{Abstract}
	\begin{abstract}
		\lipsum[10-10]
	\end{abstract}
\end{frame}


\begin{frame}{Verse}
	\begin{verse}
		Comme je descendais des Fleuves impassibles, \\
		Je ne me sentis plus guidé par les haleurs : \\
		Des Peaux-Rouges criards les avaient pris pour cibles, \\
		Les ayant cloués nus aux poteaux de couleurs.

		J’étais insoucieux de tous les équipages, \\
		Porteur de blés flamands ou de cotons anglais. \\
		Quand avec mes haleurs ont fini ces tapages, \\
		Les Fleuves m’ont laissé descendre où je voulais.
	\end{verse}
\end{frame}


\begin{frame}{Quotation}
	\begin{quotation}
		\lipsum[20-20]
	\end{quotation}
\end{frame}


\begin{frame}{Quote}
	\begin{quote}
		\lipsum[100-100]
	\end{quote}
\end{frame}


\section{Third Section}

\begin{frame}{Blocks}
	\begin{block}{Block Title}
		Now, the block content.
	\end{block}

	\begin{alertblock}{Alert Block}
		This is an alert block!
	\end{alertblock}

	\begin{exampleblock}{Example Block}
		And now an example block.
	\end{exampleblock}
\end{frame}

\begin{frame}{Proof, theorems, etc}
	\begin{theorem}
		$E = M~C^{2}$
	\end{theorem}

	\begin{lemma}
		$x \oplus y = y \oplus x$
	\end{lemma}

	\begin{proof}
		$A \cup B = (A \setminus B) \cup (B \setminus A) \cup \cap (A \cap B)$
	\end{proof}

	\begin{corollary}
		This is a corollary!
	\end{corollary}


\end{frame}


\begin{frame}{Definitions, examples, etc}

	\begin{definition}
		$\forall x, y \in S, x \prec y \equiv x^{2} \le y^{2}$
	\end{definition}

	\begin{example}
		$(a + b)^{2} = a^{2} + 2ab + b^{2}$
	\end{example}

	\begin{fact}
		This is a fact!
	\end{fact}
\end{frame}

\begin{frame}{Very long title... so long... so long ... and so long ... and so long}
	Very long title test.
\end{frame}

\end{document}
